\documentclass{beamer}

% --- Pacchetti per i font personalizzati ---
\usepackage{fontspec} % Richiede XeLaTeX o LuaLaTeX
\setsansfont{Raleway} % Font principale
\setmainfont{Raleway}
\newfontfamily\titlefont{Rokkitt} % Font per i titoli

% --- Definizione del tema ---
\usetheme{default} % Tema di base
\usecolortheme{seagull} % Colori semplici

% --- Configurazione dei colori personalizzati ---
\definecolor{linuxGreen}{RGB}{57,181,74} % Verde open source
\definecolor{linuxGray}{RGB}{85,87,83} % Grigio scuro

% --- Logo Tux in alto a sinistra ---
\setbeamertemplate{logo}{%
	\includegraphics[height=0.8cm]{images/Tux.png} % Immagine di Tux; sostituire con il percorso effettivo
}

% --- Struttura e colore delle slide ---
\setbeamercolor{frametitle}{bg=linuxGreen, fg=white}
\setbeamercolor{title}{fg=linuxGreen}
\setbeamercolor{structure}{fg=linuxGray}
\setbeamercolor{section in head/foot}{bg=linuxGreen, fg=white}
\setbeamercolor{author in head/foot}{bg=linuxGray, fg=white}

% --- Personalizzazione dei titoli con font Rokkitt ---
\setbeamerfont{title}{family=\titlefont, size=\Huge}
\setbeamerfont{frametitle}{family=\titlefont, size=\LARGE}
\setbeamerfont{section title}{family=\titlefont}

% --- Personalizzazione del titolo della slide ---
\setbeamertemplate{frametitle}{%
	\begin{beamercolorbox}[wd=\paperwidth,ht=2.5ex,dp=1ex,leftskip=0.3cm]{frametitle}
		\usebeamerfont{frametitle}\insertframetitle
	\end{beamercolorbox}
}

% --- Footer con data e titolo sezione ---
\setbeamertemplate{footline}{%
	\begin{beamercolorbox}[wd=\paperwidth,ht=2.5ex,dp=1ex,leftskip=0.3cm,rightskip=0.3cm]{section in head/foot}
		Linux Day 26 Ottobre 2024 \hfill \insertshorttitle \hfill \insertframenumber/\inserttotalframenumber
	\end{beamercolorbox}
}

% --- Struttura della presentazione ---
\title{Introduzione al Mondo Open Source}
\author{Relatore: Davide Isoardi}
\date{Linux Day 26 Ottobre 2024}
\newcommand{\license}{Creative Commons Attribution-ShareAlike 4.0 International (CC BY-SA 4.0)}

% --- Inizio del documento ---
\begin{document}
	
	% Slide di apertura
	\begin{frame}
		\titlepage
	\end{frame}
	
	% Slide dell'indice dei contenuti
	\begin{frame}{Indice dei Contenuti}
		\tableofcontents
	\end{frame}
	
	% --- Sezione Premesse ---
	\section{Premesse}
	
	\begin{frame}{Stack ISO/OSI}
		\begin{itemize}
			\item Modello di riferimento per la comunicazione in rete.
			\item Suddiviso in 7 livelli, ognuno con funzioni specifiche.
			\item Garantisce interoperabilità tra sistemi di produttori diversi.
		\end{itemize}
	\end{frame}
	
	\begin{frame}{Overview su Home Assistant}
		\begin{itemize}
			\item Piattaforma open-source per l'automazione domestica.
			\item Supporta numerosi protocolli e dispositivi IoT.
			\item Sviluppata per integrare vari sistemi in una piattaforma unica.
		\end{itemize}
	\end{frame}
	
	% --- Sezione IoT e Onde Radio ---
	\section{IoT e Onde Radio}
	
	\begin{frame}{Z-wave}
		\begin{itemize}
			\item Progettato specificamente per la domotica.
			\item Basso consumo energetico e frequenza sub-GHz.
			\item Supporta reti mesh per aumentare il raggio di copertura.
		\end{itemize}
	\end{frame}
	
	\begin{frame}{ZigBee}
		\begin{itemize}
			\item Protocollo wireless a bassa potenza, ideale per dispositivi a batteria.
			\item Ampiamente usato in domotica e automazione.
			\item Frequenza 2.4 GHz, con supporto per reti mesh.
		\end{itemize}
	\end{frame}
	
	\begin{frame}{BLE (Bluetooth Low Energy)}
		\begin{itemize}
			\item Versione ottimizzata di Bluetooth per basso consumo.
			\item Utilizzato in dispositivi come sensori e wearable.
			\item Raggio corto, ma facilmente integrabile grazie alla sua diffusione.
		\end{itemize}
	\end{frame}
	
	\begin{frame}{CoAP (Constrained Application Protocol)}
		\begin{itemize}
			\item Protocollo di comunicazione per IoT su reti a bassa larghezza di banda.
			\item Supporta trasmissione dati RESTful simile a HTTP.
			\item Ideale per dispositivi con risorse limitate.
		\end{itemize}
	\end{frame}
	
	\begin{frame}{WiFi}
		\begin{itemize}
			\item Protocollo di rete wireless diffuso e ad alta velocità.
			\item Ampia larghezza di banda e bassa latenza.
			\item Ideale per dispositivi che necessitano di velocità, ma a scapito del consumo energetico.
		\end{itemize}
	\end{frame}
	
	\begin{frame}{Matter}
		\begin{itemize}
			\item Protocollo unificato per l'interoperabilità dei dispositivi IoT.
			\item Supporta vari protocolli come WiFi, Ethernet e Thread.
			\item Sviluppato dal gruppo CSA (Connectivity Standards Alliance).
		\end{itemize}
	\end{frame}
	
	\begin{frame}{Thread}
		\begin{itemize}
			\item Protocollo di rete wireless basato su IPv6.
			\item Progettato per la comunicazione sicura tra dispositivi IoT.
			\item Supporta reti mesh, migliorando l'affidabilità e riducendo i consumi.
		\end{itemize}
	\end{frame}
	
	% --- Sezione Deadmatch ---
	\section{Deadmatch}
	
	\begin{frame}{Confronto tra i protocolli visti}
		\begin{itemize}
			\item Z-Wave e ZigBee: ideali per reti mesh a basso consumo.
			\item WiFi: maggiore velocità, ma più consumo energetico.
			\item BLE: ottimo per corto raggio e basso consumo.
			\item Matter e Thread: offrono interoperabilità e connessione sicura tra dispositivi.
		\end{itemize}
	\end{frame}
	
	\begin{frame}{Quando usare e quando evitare i protocolli}
		\begin{itemize}
			\item Z-Wave e ZigBee: perfetti per sensori e automazione domestica.
			\item WiFi: adatto per dispositivi fissi con alimentazione continua.
			\item BLE: ideale per dispositivi portatili con batteria.
			\item Matter e Thread: ottimi per integrazioni future-proof e sicurezza.
		\end{itemize}
	\end{frame}
	
	% --- Slide finale con licenza e ringraziamenti ---
	\section*{Licenza e Ringraziamenti}
	\begin{frame}{Licenza e Ringraziamenti}
		\begin{center}
			\textbf{Licenza:} \license % Testo della licenza centrato
			
			\vspace{1em}
			
			Questa presentazione è rilasciata sotto la licenza \license. Sentiti libero di condividere e adattare il contenuto rispettando i termini della licenza.
			
			\vspace{1.5em}
			
			\includegraphics[width=0.2\textwidth]{images/cc-by-sa.png} % Logo Creative Commons, ridotto e centrato
			
			\vspace{1.5em}
			
			\textbf{Ringraziamenti:} Un ringraziamento speciale all'Italian Linux Society per il supporto alla community open source.
		\end{center}
	\end{frame}
	
\end{document}
